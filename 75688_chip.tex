\documentclass[border=0.2cm]{standalone}
\usepackage{tikz}
\usepackage{circuitikz}
\usepackage{amsmath}

\renewcommand\familydefault\sfdefault
\newcommand{\PinNumber}{20}

\begin{document}
\ctikzset{logic ports=ieee}
\ctikzset{multipoles/thickness=4}
\ctikzset{multipoles/external pins thickness=2}

\begin{circuitikz}[
    chip/.style={dipchip, external pins width=0.1, external pad fraction=4}
]
    \draw (0,0) node[chip,num pins=\PinNumber] (C) {'688};

    \foreach [count=\i] \pinLabel in {
         % LEFT:
         $ \overline{\text{E}} $,
         $ \text{A0} $,
         $ \text{B0} $,
         $ \text{A1} $,
         $ \text{B1} $,
         $ \text{A2} $,
         $ \text{B2} $,
         $ \text{A3} $,
         $ \text{B3} $,
         GND,
         % RIGHT:
         $ \text{A4} $,
         $ \text{B4} $,
         $ \text{A5} $,
         $ \text{B5} $,
         $ \text{A6} $,
         $ \text{B6} $,
         $ \text{A7} $,
         $ \text{B7} $,
         $ \text{Y} $,
         $ \text{V}_{\text{CC}} $
    } {
    
        \ifnum\i<\numexpr\PinNumber/2+1\relax
            \draw (C.pin \i) node[left] {\pinLabel};
        \else
            \draw (C.pin \i) node[right] {\pinLabel};
        \fi
    }

\end{circuitikz}

\end{document}
