\documentclass[border=0.2cm]{standalone}
\usepackage{tikz}
\usepackage{circuitikz}
\usepackage{amsmath}

\renewcommand\familydefault\sfdefault
\newcommand{\PinNumber}{16}

\begin{document}

\ctikzset{logic ports=ieee}
\ctikzset{multipoles/thickness=4}
\ctikzset{multipoles/external pins thickness=2}

\begin{circuitikz}[
    chip/.style={dipchip, external pins width=0.1, external pad fraction=4}
]
    \draw (0,0) node[chip,num pins=\PinNumber,] (C) {'595};

    \foreach [count=\i] \pinLabel in {
         % LEFT:
         $\text{Q}_{\text{B}}$,
         $\text{Q}_{\text{C}}$, 
         $\text{Q}_{\text{D}}$, 
         $\text{Q}_{\text{E}}$,
         $\text{Q}_{\text{F}}$,
         $\text{Q}_{\text{G}}$,
         $\text{Q}_{\text{H}}$,
         GND,
         % RIGHT:
         $\text{Q}_{\text{H'}}$,
         $\overline{\text{SRCLR}}$,
         SRCLK,
         RCLK,
         $\overline{\text{OE}}$,
         SER,
         $\text{Q}_{\text{A}}$,
         $\text{V}_{\text{CC}}$
    } {
    
        \ifnum\i<\numexpr\PinNumber/2+1\relax
            \draw (C.pin \i) node[left] {\pinLabel};
        \else
            \draw (C.pin \i) node[right] {\pinLabel};
        \fi
    }
\end{circuitikz}

\end{document}
