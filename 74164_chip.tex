\documentclass[border=0.2cm]{standalone}
\usepackage{tikz}
\usepackage{circuitikz}
\usepackage{amsmath}

\renewcommand\familydefault\sfdefault
\newcommand{\PinNumber}{14}

\begin{document}

\begin{circuitikz}[
    logic ports=ieee,
    multipoles/thickness=4,
    multipoles/external pins thickness=2
]
    \draw (0,0) node[dipchip,num pins=\PinNumber, external pins width=0.1, external pad fraction=4 ] (C) {'166};
    %\node [right, font=\tiny] at (C.bpin 1) {RST};

    \foreach [count=\i] \pinLabel in {A, B, $\text{Q}_{\text{A}}$, $\text{Q}_{\text{B}}$, $\text{Q}_{\text{C}}$, $\text{Q}_{\text{D}}$, GND, CLK, $\overline{\text{CLR}}$, $\text{Q}_{\text{E}}$ ,$\text{Q}_{\text{F}}$, $\text{Q}_{\text{G}}$, $\text{Q}_{\text{H}}$, $\text{V}_{\text{CC}}$} {
    
        \ifnum\i<\numexpr\PinNumber/2+1\relax
            \draw (C.pin \i) node[left] {\pinLabel};
        \else
            \draw (C.pin \i) node[right] {\pinLabel};
        \fi
    }
\end{circuitikz}

\end{document}
